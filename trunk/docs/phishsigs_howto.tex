%% LyX 1.5.3 created this file.  For more info, see http://www.lyx.org/.
%% Do not edit unless you really know what you are doing.
\documentclass[a4paper,english,10pt]{article}
\usepackage{amssymb}
\usepackage{pslatex}
\usepackage[T1]{fontenc}
\usepackage[dvips]{graphicx}
\usepackage{url}
\usepackage{fancyhdr}
\usepackage{varioref}
\usepackage{prettyref}
\date{}

\begin{document}

\title{{\huge Phishing signatures creation HOWTO}}
\author{T\"or\"ok Edwin}
\maketitle

%TODO: define a LaTeX command, instead of using \textsc{RealURL} each time

\section{Database file format}

\subsection{PDB format}
This file contains urls/hosts that are target of phishing attempts.
It contains lines in the following format:
\begin{verbatim}
R[Filter]:RealURL:DisplayedURL[:FuncLevelSpec]
H[Filter]:DisplayedHostname[:FuncLevelSpec]
\end{verbatim}

\begin{description}
 \item [{R}] regular expression, for the concatenated URL
 \item [{H}] matches the \verb+DisplayedHostname+ as a simple pattern (literally, no regular expression)
 	\begin{itemize}
 		\item the pattern can match either the full hostname
 		\item or a subdomain of the specified hostname
 		\item to avoid false matches in case of subdomain matches, the engine checks that there  is a dot(\verb+.+) or a space(\verb+ +) before the matched portion
	\end{itemize}
 \item [{Filter}] an (optional) 3-digit hexadecimal number representing flags that should be filtered.
	\begin{itemize}
 		\item flag filtering only makes sense in .pdb files. (however clamav won't complain if you put flags in .wdb files, it will just skip them)
 		\item for details on how to construct a flag number see section \prettyref{sec:Flags}
	\end{itemize}

 \item [{\textsc{RealURL}}] is the URL the user is sent to, example: \emph{href} attribute of an html anchor (\emph{<a> tag})
 \item [{\textsc{DisplayedURL}}] is the URL description displayed to the user, where its \emph{claimed} they are sent, example: contents of an html anchor (\emph{<a> tag})
 \item [{DisplayedHostname}] is the hostname portion of the \textsc{DisplayedURL}
 \item [{FuncLevelSpec}] an (optional) functionality level, 2 formats are possible:
	\begin{itemize}
 		\item \verb+minlevel+ all engines having functionality level >= \verb+minlevel+ will load this line
 		\item \verb+minlevel-maxlevel+ engines with functionality level $>= $ \verb+minlevel+, and $< $ \verb+maxlevel+ will load this line
	\end{itemize}
\end{description}

\subsection{WDB format}
This file contains whitelisted url pairs
It contains lines in the following format:
\begin{verbatim}
X:RealURL:DisplayedURL[:FuncLevelSpec]
M:RealHostname:DisplayedHostname[:FuncLevelSpec]
\end{verbatim}

\begin{description}
 \item [{X}] regular expression, for the \emph{entire URL}, not just the hostname
 \begin{itemize}
  \item The regular expression is by default anchored to start-of-line and end-of-line, as if you have used \verb+^RegularExpression$+
  \item A trailing \verb+/+ is automatically added both to the regex, and the input string to avoid false matches
  \item The regular expression matches the \emph{concatenation} of the \textsc{RealURL}, a colon(\verb+:+), and the \textsc{DisplayedURL} as a single string. It doesn't separately match \textsc{RealURL} and \textsc{DisplayedURL}!
 \end{itemize}
 \item [{M}] matches hostname, or subdomain of it, see notes for {H} above
\end{description}

\subsection{Hints}

\begin{itemize}
 \item empty lines are ignored
 \item the colons are mandatory
 \item Don't leave extra spaces on the end of a line!
 \item if any of the lines don't conform to this format, clamav will abort with a Malformed Database Error
 \item see section \vref{sub:Extraction-of-realURL,} for more details on \textsc{realURL/displayedURL}
\end{itemize}


\subsection{Examples of PDB signatures}
To check for phishing mails that target amazon.com, or subdomains of amazon.com:
\begin{verbatim}
H:amazon.com
\end{verbatim}

To do the same, but for amazon.co.uk:
\begin{verbatim}
H:amazon.co.uk 
\end{verbatim}

To limit the signatures to certain engine versions:
\begin{verbatim}
H:amazon.co.uk:20-30
H:amazon.co.uk:20-
H:amazon.co.uk:0-20 
\end{verbatim}
First line: engine versions 20, 21, ..., 29 can load it 

Second line: engine versions >= 20 can load it

Third line: engine versions < 20 can load it

In a real situation, you'd probably use the second form. A situation like that would be if you are using a feature of the signatures
not available in earlier versions, or if earlier versions have bugs with your signature. Its neither case here, the above examples
are for illustrative purposes only.

\subsection{Examples of WDB signatures}
To allow amazon's country specific domains and amazon.com, to mix domain names in \textsc{DisplayedURL}, and \textsc{RealURL}:
\begin{verbatim}
X:.+\.amazon\.(at|ca|co\.uk|co\.jp|de|fr)([/?].*)?:.+\.amazon\.com([/?].*)?:17-
\end{verbatim}
Explanation of this signature:
\begin{description}
 \item [{X:}] this is a regular expression
 \item [{:17-}] load signature only for engines with functionality level >= 17 (recommended for type X)
\end{description}

The regular expression is the following (X:, :17- stripped, and a / appended)
\begin{verbatim}
.+\.amazon\.(at|ca|co\.uk|co\.jp|de|fr)([/?].*)?:.+\.amazon\.com([/?].*)?/ 
\end{verbatim}

Explanation of this regular expression (note that it is a single regular expression, and not 2 regular
expressions splitted at the {:}).
\begin{itemize}
 \item \verb;.+; any subdomain of
 \item \verb;\.amazon\.; domain we are whitelisting (\textsc{RealURL} part)
 \item \verb;(at|ca|co\.uk|co\.jp|de|fr); country-domains: at, ca, co.uk, co.jp, de, fr
 \item \verb;([/?].*)?; recomended way to end real url part of whitelist, this protects against embedded URLs (evilurl.example.com/amazon.co.uk/)
 \item \verb;:; \textsc{RealURL} and \textsc{DisplayedURL} are concatenated via a {:}, so match a literal {:} here
 \item \verb;.+; any subdomain of
 \item \verb;\.amazon\.com; whitelisted DisplayedURL
 \item \verb;([/?].*)?; recommended way to end displayed url part, to protect against embedded URLs
 \item \verb;/; automatically added to further protect against embedded URLs
\end{itemize}

When you whitelist an entry make sure you check that both domains are owned by the same entity.
What this whitelist entry allows is:
Links claiming to point to amazon.com (\textsc{DisplayedURL}), but really go to country-specific domain of amazon (\textsc{RealURL}).

\subsection{Example for how the URL extractor works}
Consider the following HTML file:
\begin{verbatim}
<html>
<a href="http://1.realurl.example.com/">
  1.displayedurl.example.com
</a>
<a href="http://2.realurl.example.com">
  2 d<b>i<p>splayedurl.e</b>xa<i>mple.com
</a>
<a href="http://3.realurl.example.com">	
  3.nested.example.com
  <a href="http://4.realurl.example.com">
    4.displayedurl.example.com
  </a>
</a>
<form action="http://5.realurl.example.com">
  sometext
  <img src="http://5.displayedurl.example.com/img0.gif"/>
  <a href="http://5.form.nested.displayedurl.example.com">
    5.form.nested.link-displayedurl.example.com
  </a>
</form>
<a href="http://6.realurl.example.com">
  6.displ
  <img src="6.displayedurl.example.com/img1.gif"/>
  ayedurl.example.com
</a>
<a href="http://7.realurl.example.com">
  <iframe src="http://7.displayedurl.example.com">
</a>
\end{verbatim}

The phishing engine extract the following \textsc{RealURL/DisplayedURL} pairs from it:
\begin{verbatim}
http://1.realurl.example.com/
1.displayedurl.example.com

http://2.realurl.example.com
2displayedurl.example.com

http://3.realurl.example.com
3.nested.example.com

http://4.realurl.example.com
4.displayedurl.example.com

http://5.realurl.example.com
http://5.displayedurl.example.com/img0.gif

http://5.realurl.example.com
http://5.form.nested.displayedurl.example.com

http://5.form.nested.displayedurl.example.com
5.form.nested.link-displayedurl.example.com

http://6.realurl.example.com
6.displayedurl.example.com

http://6.realurl.example.com
6.displayedurl.example.com/img1.gif
\end{verbatim}


\subsection{How matching works}

\subsubsection{RealURL, displayedURL concatenation\label{sub:RealURL,-displayedURL-concatenation}}

The phishing detection module processes pairs of \textsc{RealURL/DisplayedURL}.
Matching against daily.wdb is done as follows: the \textsc{realURL} is concatenated with a \verb+:+, and with the \textsc{DisplayedURL}, then that \emph{line} is matched against the lines in daily.wdb/daily.pdb

So if you have this line in daily.wdb:
\begin{verbatim}
M:www.google.ro:www.google.com
\end{verbatim}

and this href: \verb+<a href='http://www.google.ro'>www.google.com</a>+
then it will be whitelisted, but: \verb+<a href='http://images.google.com'>www.google.com</a>+
will not.

\subsubsection{What happens when a match is found}

In the case of the whitelist, a match means that the \textsc{RealURL/DisplayedURL}
combination is considered \textsc{clean}, and no further checks are
performed on it.

In the case of the domainlist, a match means that the \textsc{RealURL/displayedURL}
is going to be checked for phishing attempts. 

Furthermore you can restrict what checks are to be performed by specifying the 3-digit hexnumber.

\subsubsection{Extraction of \textsc{realURL}, \textsc{displayedURL} from HTML tags\label{sub:Extraction-of-realURL,}}

The html parser extracts pairs of \textsc{realURL}/\textsc{displayedURL}
based on the following rules.

In version 0.93: After URLs have been extracted, they are normalized, and cut after the hostname.
\verb+http://test.example.com/path/somecgi?queryparameters+ becomes \verb+http://test.example.com/+

\begin{description}
\item [{a}] (anchor) the \emph{href} is the \textsc{realURL}, its \emph{contents}
is the \textsc{displayedURL}

\begin{description}
\item [{contents}] is the tag-stripped contents of the <a> tags, so for
example <b> tags are stripped (but not their contents)
\end{description}
nesting another <a> tag withing an <a> tag (besides being invalid
html) is treated as a </a><a..

\item [{form}] the \emph{action} attribute is the \textsc{realURL}, and a
nested <a> tag is the \textsc{displayedURL}
\item [{img/area}] if nested within an \emph{<a>} tag, the \textsc{realURL}
is the \emph{href} of the a tag, and the \emph{src/dynsrc/area} is
the \textsc{displayedURL} of the img 


if nested withing a \emph{form} tag, then the action attribute of
the \emph{form} tag is the \textsc{realURL} 

\item [{iframe}] if nested withing an \emph{<a>} tag the \emph{src} attribute
is the \textsc{displayedURL}, and the \emph{href} of its parent \emph{a} tag
is the \textsc{realURL}


if nested withing a \emph{form} tag, then the action attribute of
the \emph{form} tag is the \textsc{realURL}

\end{description}

\subsubsection{Example}

Consider this html file:

\begin{quote}
\emph{<a href=''evilurl''>www.paypal.com</a>}

\emph{<a href=''evilurl2'' title=''www.ebay.com''>click here to
sign in</a>}

\emph{<form action=''evilurl\_form''>}

\emph{Please sign in to <a href=''cgi.ebay.com''>Ebay</a> using
this form}

\emph{<input type='text' name='username'>Username</input>}

\emph{....}

\emph{</form>}

\emph{<a href=''evilurl''><img src=''images.paypal.com/secure.jpg''></a>}
\end{quote}
The resulting \textsc{realURL/displayedURL} pairs will be (note that
one tag can generate multiple pairs):

\begin{itemize}
\item evilurl / www.paypal.com
\item evilurl2 / click here to sign in
\item evilurl2 / www.ebay.com
\item evilurl\_form / cgi.ebay.com
\item cgi.ebay.com / Ebay
\item evilurl / image.paypal.com/secure.jpg
\end{itemize}

\subsection{Simple patterns\label{sec:Simple-patterns}}

Simple patterns are matched literally, i.e. if you say: 

\begin{quote}
www.google.com
\end{quote}
it is going to match \emph{www.google.com}, and only that. The \emph{.
(dot)} character has no special meaning (see the section on regexes
\vref{sec:Regular-expressions} for how the \emph{.(dot)} character
behaves there)


\subsection{Regular expressions\label{sec:Regular-expressions}}

POSIX regular expressions are supported, and you can consider that
internally it is wrapped by \emph{\textasciicircum{}}, and \emph{\$.}
In other words, this means that the regular expression has to match
the entire concatenated (see section \vref{sub:RealURL,-displayedURL-concatenation}
for details on concatenation) url.

It is recomended that you read section \vref{sec:Introduction-to-regular}
to learn how to write regular expressions, and then come back and
read this for hints.

Be advised that clamav contains an internal, very basic regex matcher
to reduce the load on the regex matching core. Thus it is recomended
that you avoid using regex syntax not supported by it at the very
beginning of regexes (at least the first few characters).

Currently the clamav regex matcher supports:

\begin{itemize}
\item . (dot) character
\item $\backslash$ (escaping special characters)
\item | (pipe) alternatives
\item {[}] (character classes)
\item () (paranthesis for grouping, but no group extraction is performed)
\item other non-special characters
\end{itemize}
Thus the following are not supported:

\begin{itemize}
\item + repetition
\item {*} repetition
\item \{\} repetition
\item backreferences
\item lookaround
\item other {}``advanced'' features not listed in the supported list ;)
\end{itemize}
This however shouldn't discourage you from using the {}``not directly
supported features {}``, because if the internal engine encounters
unsupported syntax, it passes it on to the POSIX regex core (beginning
from the first unsupported token, everything before that is still
processed by the internal matcher). An example might make this more
clear:

\emph{www$\backslash$.google$\backslash$.(com|ro|it) ({[}a-zA-Z])+$\backslash$.google$\backslash$.(com|ro|it)}

Everything till \emph{({[}a-zA-Z])+} is processed internally, that
paranthesis (and everything beyond) is processed by the posix core.

Examples of url pairs that match: 

\begin{itemize}
\item \emph{www.google.ro images.google.ro}
\item www.google.com images.google.ro
\end{itemize}
Example of url pairs that don't match:

\begin{itemize}
\item www.google.ro images1.google.ro
\item images.google.com image.google.com
\end{itemize}

\subsection{Flags\label{sec:Flags}}

Flags are a binary OR of the following numbers:

\begin{description}
\item [{HOST\_SUFFICIENT}] 1
\item [{DOMAIN\_SUFFICIENT}] 2
\item [{DO\_REVERSE\_LOOKUP}] 4
\item [{CHECK\_REDIR}] 8
\item [{CHECK\_SSL}] 16 
\item [{CHECK\_CLOAKING}] 32
\item [{CLEANUP\_URL}] 64 
\item [{CHECK\_DOMAIN\_REVERSE}] 128 
\item [{CHECK\_IMG\_URL}] 256 
\item [{DOMAINLIST\_REQUIRED}] 512 
\end{description}
The names of the constants are self-explanatory.

These constants are defined in libclamav/phishcheck.h, you can check
there for the latest flags.

There is a default set of flags that are enabled, these are currently:
\begin{verbatim}
(CLEANUP\_URL|CHECK\_SSL|CHECK\_CLOAKING|CHECK\_IMG\_URL)
\end{verbatim}
ssl checking is performed only for a tags currently.

You must decide for each line in the domainlist if you want to filter
any flags (that is you don't want certain checks to be done), and
then calculate the binary OR of those constants, and then convert
it into a 3-digit hexnumber. For example you devide that domain\_sufficient
shouldn't be used for ebay.com, and you don't want to check images
either, so you come up with this flag number: $2|256\Rightarrow$258$(decimal)\Rightarrow102(hexadecimal)$

So you add this line to daily.wdb:

\begin{itemize}
\item R102~www.ebay.com~.+
\end{itemize}

\section{Introduction to regular expressions\label{sec:Introduction-to-regular}}

Recomended reading:

\begin{itemize}
\item http://www.regular-expressions.info/quickstart.html
\item http://www.regular-expressions.info/tutorial.html
\item regex(7) man-page: http://www.tin.org/bin/man.cgi?section=7\&topic=regex
\end{itemize}

\subsection{Special characters}

\begin{description}
\item [{{[}}] the opening square bracket - it marks the beginning of a
character class, see section\vref{sub:Character-classes}
\item [{$\backslash$}] the backslash - escapes special characters,
see section \vref{sub:Escaping}
\item [{\^{ }}] the caret - matches the beginning of a line (not needed
in clamav regexes, this is implied)
\item [{\$}] the dollar sign - matches the end of a line (not needed in
clamav regexes, this is implied)
\item [{\.{ }}] the period or dot - matches \emph{any} character
\item [{|}] the vertical bar or pipe symbol - matches either of the token
on its left and right side, see section\vref{sub:Alternation}
\item [{?}] the question mark - matches optionally the left-side token,
see section\vref{sub:Optional-matching,-and}
\item [{{*}}] the asterisk or star - matches 0 or more occurences of the
left-side token, see section \vref{sub:Optional-matching,-and}
\item [{+}] the plus sign - matches 1 or more occurences of the left-side
token, see section \vref{sub:Optional-matching,-and}
\item [{(}] the opening round bracket - \c{m}arks beginning of a group,
see section \vref{sub:Groups}
\item [{)}] the closing round bracket - marks end of a group, see section\vref{sub:Groups}
\end{description}

\subsection{Character classes\label{sub:Character-classes}}


\subsection{Escaping\label{sub:Escaping}}

Escaping has two purposes: 

\begin{itemize}
\item it allows you to actually match the special characters themselves,
for example to match the literal \emph{+}, you would write \emph{$\backslash$+}
\item it also allows you to match non-printable characters, such as the
tab (\emph{$\backslash$t}), newline (\emph{$\backslash$n}),
..
\end{itemize}
However since non-printable characters are not valid inside an url,
you won't have a reason to use them.


\subsection{Alternation\label{sub:Alternation}}


\subsection{Optional matching, and repetition\label{sub:Optional-matching,-and}}


\subsection{Groups\label{sub:Groups}}

Groups are usually used together with repetition, or alternation.
For example: \emph{(com|it)+} means: match 1 or more repetitions of
\emph{com} or \emph{it,} that is it matches: com, it, comcom, comcomcom,
comit, itit, ititcom,... you get the idea.

Groups can also be used to extract substring, but this is not supported
by the clam engine, and not needed either in this case.


\section{How to create database files}


\subsection{How to create and maintain the whitelist (daily.wdb)}

If the phishing code claims that a certain mail is phishing, but its
not, you have 2 choices:

\begin{itemize}
\item examine your rules daily.pdb, and fix them if necessary (see: section\vref{sub:How-to-create})
\item add it to the whitelist (discussed here)
\end{itemize}
Lets assume you are having problems because of links like this in
a mail:

\begin{verbatim}
<a href=''http://69.0.241.57/bCentral/L.asp?L=XXXXXXXX''>
  http://www.bcentral.it/
</a>
\end{verbatim}
After investigating those sites further, you decide they are no threat,
and create a line like this in daily.wdb:

\begin{quote}
R http://www$\backslash$.bcentral$\backslash$.it/.+ http://69$\backslash$.0$\backslash$.241$\backslash$.57/bCentral/L$\backslash$.asp?L=.+ 
\end{quote}
Note: urls like the above can be used to track unique mail recipients,
and thus know if somebody actually reads mails (so they can send more
spam). However since this site required no authentication information,
it is safe from a phishing point of view.


\subsection{How to create and maintain the domainlist (daily.pdb)\label{sub:How-to-create}}

When not using --phish-scan-alldomains (production environments for
example), you need to decide which urls you are going to check. 

Although at a first glance it might seem a good idea to check everything,
it would produce false positives. Particularly newsletters, ads, etc.
are likely to use URLs that look like phishing attempts.

Lets assume that you've recently seen many phishing attempts claiming
they come from Paypal. Thus you need to add paypal to daily.pdb:

\begin{quote}
R .+ .+$\backslash$.paypal$\backslash$.com
\end{quote}
The above line will block (detect as phishing) mails that contain
urls that claim to lead to paypal, but they don't in fact.

Be carefull not to create regexes that match a too broad range of
urls though.


\subsection{Dealing with false positives, and undetected phishing mails}


\subsubsection{False positives}

Whenever you see a false positive (mail that is detected as phishing,
but its not), you need to examine \emph{why} clamav decided that its
phishing. You can do this easily by building clamav with debugging
(./configure --enable-experimental --enable-debug), and then running
a tool:

\begin{quote}
\$contrib/phishing/why.py phishing.eml
\end{quote}
This will show the url that triggers the phish verdict, and a reason
why that url is considered phishing attempt.

Once you know the reason, you might need to modify daily.pdb (if one
of yours rules inthere are too broad), or you need to add the url
to daily.wdb. If you think the algorithm is incorrect, please file
a bugreport on bugzilla.clamav.net, including the output of \emph{why.py}.


\subsubsection{Undetected phish mails}

Using why.py doesn't help here unfortunately (it will say: clean),
so all you can do is:

\begin{quote}
\$clamscan/clamscan --phish-scan-alldomains undetected.eml
\end{quote}
And see if the mail is detected, if yes, then you need to add an appropiate
line to daily.pdb (see section \vref{sub:How-to-create}).

If the mail is not detected, then try using:

\begin{quote}
\$clamscan/clamscan --debug undetected.eml|less
\end{quote}

Then see what urls are being checked, see if any of them is in a
whitelist, see if all urls are detected, etc.


\end{document}
