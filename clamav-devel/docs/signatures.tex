\documentclass[a4paper,titlepage,12pt]{article}
\usepackage{amssymb}
\usepackage{pslatex}
\usepackage[dvips]{graphicx}
\usepackage{wrapfig}
\usepackage{url}
\date{}

\begin{document}

    \begin{center}
	\huge Creating signatures for ClamAV\\
	\vspace{2cm}
    \end{center}

    \noindent
    \section{Introduction}
    CVD (ClamAV Virus Database) is a digitally signed tarball file that
    contains one or more databases. The header is a 512 bytes long string
    with colon separated fields:
    \begin{verbatim}
ClamAV-VDB:build time:version:number of signatures:functionality
level required:MD5 checksum:digital signature:builder name:build time (sec)
    \end{verbatim}
    \verb+sigtool --info+ displays detailed information about a CVD file:
    \begin{verbatim}
zolw@localhost:/usr/local/share/clamav$ sigtool -i main.cvd
Build time: 09 Jun 2006 22-19 +0200
Version: 39
# of signatures: 58116
Functionality level: 8
Builder: tkojm
MD5: a9a400e70dcbfe2c9e11d78416e1c0cc
Digital signature: 0s12V8OxLWO95fNNv+kTxj7CEWBW/1TKOGC7G4RelhogruBYw8dJeIX2+yhxex/XsLohxoEuXxC2CaFXiiTbrbvpK2USIxkpn53n6LYVV6jKgkP5sa08MdJE7cl29H1slfCrdaevBUZ1Z/UefkRnV6p3iQVpDPsBwqFRbrem33b
Verification OK.
    \end{verbatim}
    There are two CVD databases in ClamAV: \emph{main.cvd} and \emph{daily.cvd}
    for daily updates.

    \section{Signature format}

    \subsection{MD5}
    There's an easy way to create signatures for static malware using MD5
    checksums. To create a signature for \verb+test.exe+ use the \verb+--md5+
    option of sigtool:
    \begin{verbatim}
zolw@localhost:/tmp/test$ sigtool --md5 test.exe > test.hdb
zolw@localhost:/tmp/test$ cat test.hdb 
48c4533230e1ae1c118c741c0db19dfb:17387:test.exe
    \end{verbatim}
    That's it! The signature is ready to use:
    \begin{verbatim}
zolw@localhost:/tmp/test$ clamscan -d test.hdb test.exe 
test.exe: test.exe FOUND

----------- SCAN SUMMARY -----------
Known viruses: 1
Scanned directories: 0
Engine version: 0.88.2
Scanned files: 1
Infected files: 1
Data scanned: 0.02 MB
Time: 0.024 sec (0 m 0 s)
    \end{verbatim}
    You can edit it to change the name (by default sigtool uses the file name).
    Remember that all MD5 signatures must be placed inside \verb+*.hdb+ files
    and you can include any number of signatures inside a single file. To get
    them automatically loaded every time clamscan/clamd starts just copy them
    to the local virus database directory.

    \subsection{MD5, PE section based}
    You can create an MD5 signature for a specific section in a PE file.
    Such signatures are stored in .mdb files in the following format:
    \begin{verbatim}
PESectionSize:MD5:MalwareName
    \end{verbatim}

    \subsection{Hexadecimal signatures}
    ClamAV keeps viral fragments in hexadecimal format. If you don't know how
    to get a proper signature please try the MD5 method or submit your sample
    at \url{http://www.clamav.net/sendvirus.html}

    \subsubsection{Hexadecimal format}
    You can use \verb+sigtool --hex-dump+ to convert arbitrary data into
    hexadecimal format:
    \begin{verbatim}
zolw@localhost:/tmp/test$ sigtool --hex-dump
How do I look in hex?
486f7720646f2049206c6f6f6b20696e206865783f0a
    \end{verbatim}

    \subsubsection{Wildcards}
    ClamAV supports the following extensions inside hex signatures:
    \begin{itemize}
	\item \verb+??+\\
	Match any byte.
	\item \verb+*+\\
	Match any number of bytes.
	\item \verb+{n}+\\
	Match n bytes.
	\item \verb+{-n}+\\
	Match n or less bytes.
	\item \verb+{n-}+\\
	Match n or more bytes.
	\item \verb+(a|b)+\\
	Match a and b (you can use more alternate characters).
    \end{itemize}

    \subsubsection{Basic signature format}
    The simplest signatures are of the format:
    \begin{verbatim}
MalwareName=HexSignature
    \end{verbatim}
    ClamAV will analyse a whole content of a file trying to match it. All
    signatures of this type must be placed in \verb+*.db+ files.

    \subsubsection{Extended signature format}
    Extended signature format allows on including additional information about
    target file type, virus offset and required engine version.
    The format is:
    \begin{verbatim}
MalwareName:TargetType:Offset:HexSignature[:MinEngineFunctionalityLevel:[Max]]
    \end{verbatim}
    where \verb+TargetType+ is one of the following decimal numbers describing
    the target file type:
    \begin{itemize}
	\item 0 = any file
	\item 1 = Portable Executable
	\item 2 = OLE2 component (e.g. VBA script)
	\item 3 = HTML (normalised)
	\item 4 = Mail file
	\item 5 = Graphics (to help catching exploits in JPEG files)
	\item 6 = ELF
    \end{itemize}
    And	\verb+Offset+ is an asterisk or a decimal number \verb+n+ possibly
    combined with a special string:
    \begin{itemize}
	\item \verb+*+ = any
	\item \verb+n+ = absolute offset
	\item \verb+EOF-n+ = end of file minus \verb+n+ bytes
    \end{itemize}
    Signatures for Portable Executables files (target = 1) also support:
    \begin{itemize}
	\item \verb#EP+n# = entry point plus n bytes (\verb#EP+0# if you
	want to anchor to \verb+EP+)
	\item \verb#EP-n# = entry point minus n bytes
	\item \verb#Sx+n# = start of section's \verb+x+ (counted from 0)
	data plus \verb+n+ bytes
	\item \verb#Sx+n# = start of section's \verb+x+ data minus \verb+n+ bytes
	\item \verb#SL+n# = start of last section plus \verb+n+ bytes
	\item \verb#SL-n# = start of last section minux \verb+n+ bytes
    \end{itemize}
    All signatures in the extended format must be placed in \verb+*.ndb+ files.

    \subsection{Signatures based on archive metadata}
    In order to detect some malware which spreads inside of Zip or RAR archives
    (especially encrypted ones) you can try to create a signature describing
    a malicious archived file. The general format is:
\begin{verbatim}
virname:encrypted:filename:normal size:csize:crc32:cmethod:fileno:max depth
\end{verbatim}
    \begin{itemize}
	\item Virus name
	\item Encryption flag (1 -- encrypted, 0 -- not encrypted)
	\item File name (* to ignore)
	\item Normal (uncompressed) size (* to ignore)
	\item Compressed size (* to ignore)
	\item CRC32 (* to ignore)
	\item Compression method (* to ignore)
	\item File position in archive (* to ignore)
	\item Maximum number of nested archives (* to ignore)
    \end{itemize}
    The database should have the extension \verb+.zmd+ or \verb+.rmd+ for
    Zip or RAR archive respectively.

    \subsection{Whitelist database}
    To whitelist a specific file use the MD5 signature format and place
    it in the database with the extension \verb+.fp+.

    \subsection{Signature names}
    ClamAV uses the following prefixes for particular malware:
    \begin{itemize}
	\item \emph{Worm} for Internet worms
	\item \emph{Trojan} for backdoor programs
	\item \emph{Adware} for adware
	\item \emph{Flooder} for flooders
        \item \emph{HTML} for HTML files
        \item \emph{Email} for email messages
        \item \emph{IRC} for IRC trojans
	\item \emph{JS} for Java Script malware
	\item \emph{PHP} for PHP malware
	\item \emph{ASP} for ASP malware
	\item \emph{VBS} for VBS malware
	\item \emph{BAT} for BAT malware
	\item \emph{W97M}, \emph{W2000M} for Word macro viruses
	\item \emph{X97M}, \emph{X2000M} for Excel macro viruses
	\item \emph{O97M}, \emph{O2000M} for general Office macro viruses
	\item \emph{DoS} for Denial of Service attack software
	\item \emph{DOS} for old DOS malware
	\item \emph{Exploit} for popular exploits
	\item \emph{VirTool} for virus construction kits
	\item \emph{Dialer} for dialers
	\item \emph{Joke} for hoaxes
    \end{itemize}
    Important rules of the naming convention:
    \begin{itemize}
	\item always use a -zippwd postfix in the malware name for signatures of	      type zmd,
	\item always use a -rarpwd postfix in the malware name for signatures
	      of type rmd,
	\item only use alphanumeric characters, dash (-), dot (.), underscores
	      (\_) in malware names, never use space, apostrophe or quote mark.
    \end{itemize}

    \section{Special files}

    \subsection{HTML}
    ClamAV contains a special HTML normalisation code required to detect
    HTML exploits. Running \verb+sigtool --html-normalise+ on a HTML file
    should create the following files:
    \begin{itemize}
	\item comment.html - the whole file normalised
	\item nocomment.html - the file normalised, with all comments removed
	\item script.html - the parts of the file in \verb+<script>+ tags
	      (lowercased)
    \end{itemize}
    The code automatically decodes JScript.encode parts and char ref's (e.g.
    \verb+&#102;+). You need to create a signature against one of the created
    files. To eliminate potential false positive alerts you should use
    extended signature format with target type of 3.

    \subsection{Compressed Portable Executable files}
    If the file is compressed with UPX, FSG, Petite or other executable packer
    (supported by libclamav) run \verb+clamscan+ with
    \verb+--debug --leave-temps+. Example output on FSG compressed file:
    \begin{verbatim}
LibClamAV debug: UPX/FSG: empty section found - assuming compression
LibClamAV debug: FSG: found old EP @1554
LibClamAV debug: FSG: Successfully decompressed
LibClamAV debug: UPX/FSG: Decompressed data saved in /tmp/clamav-4eba73ff4050a26
    \end{verbatim}
    and then create a signature for \verb+/tmp/clamav-4eba73ff4050a26+

\end{document}
