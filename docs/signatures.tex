\documentclass[a4paper,titlepage,12pt]{article}
\usepackage{amssymb}
\usepackage{pslatex}
\usepackage[dvips]{graphicx}
\usepackage{wrapfig}
\usepackage{url}
\date{}

\begin{document}

    \begin{center}
	\huge Creating signatures for ClamAV\\
	\vspace{2cm}
    \end{center}

    \noindent
    \section{Introduction}
    CVD (ClamAV Virus Database) is a digitally signed container that
    includes signature databases in various text formats. The header
    of the container is a 512 bytes long string with colon separated fields:
    \begin{verbatim}
ClamAV-VDB:build time:version:number of signatures:functionality
level required:MD5 checksum:digital signature:builder name:build
time (sec)
    \end{verbatim}
    \verb+sigtool --info+ displays detailed information about a given CVD file:
    \begin{verbatim}
zolw@localhost:/usr/local/share/clamav$ sigtool -i main.cvd
File: main.cvd
Build time: 09 Dec 2007 15:50 +0000
Version: 45
Signatures: 169676
Functionality level: 21
Builder: sven
MD5: b35429d8d5d60368eea9630062f7c75a
Digital signature: dxsusO/HWP3/GAA7VuZpxYwVsE9b+tCk+tPN6OyjVF/U8
JVh4vYmW8mZ62ZHYMlM903TMZFg5hZIxcjQB3SX0TapdF1SFNzoWjsyH53eXvMDY
eaPVNe2ccXLfEegoda4xU2TezbGfbSEGoU1qolyQYLX674sNA2Ni6l6/CEKYYh
Verification OK.
    \end{verbatim}
    The ClamAV project distributes two CVD files: \emph{main.cvd} and
    \emph{daily.cvd}.

    \section{Signature formats}

    \subsection{MD5}
    The easiest way to create signatures for ClamAV is to use MD5 checksums,
    however this method can be only used against static malware. To create
    a signature for \verb+test.exe+ use the \verb+--md5+ option of sigtool:
    \begin{verbatim}
zolw@localhost:/tmp/test$ sigtool --md5 test.exe > test.hdb
zolw@localhost:/tmp/test$ cat test.hdb 
48c4533230e1ae1c118c741c0db19dfb:17387:test.exe
    \end{verbatim}
    That's it! The signature is ready to use:
    \begin{verbatim}
zolw@localhost:/tmp/test$ clamscan -d test.hdb test.exe 
test.exe: test.exe FOUND

----------- SCAN SUMMARY -----------
Known viruses: 1
Scanned directories: 0
Engine version: 0.92.1
Scanned files: 1
Infected files: 1
Data scanned: 0.02 MB
Time: 0.024 sec (0 m 0 s)
    \end{verbatim}
    You can change the name (by default sigtool uses the name of the file)
    and place it inside a \verb+*.hdb+ file. A single database file can
    include any number of signatures. To get them automatically loaded
    each time clamscan/clamd starts just copy the database file(s) into
    the local virus database directory (eg. /usr/local/share/clamav).

    \subsection{MD5, PE section based}
    You can create a MD5 signature for a specific section in a PE file.
    Such signatures shall be stored inside \verb+.mdb+ files in the
    following format:
    \begin{verbatim}
PESectionSize:MD5:MalwareName
    \end{verbatim}
    The easiest way to generate MD5 based section signatures is to extract
    target PE sections into separate files and then run sigtool with the
    option \verb+--mdb+

    \subsection{Hexadecimal signatures}
    ClamAV stores all signatures in a hexadecimal format. By a hex-signature
    here we mean a fragment of a malware's body converted into a hexadecimal
    string which can be additionally extended with various wildcards.

    \subsubsection{Hexadecimal format}
    You can use \verb+sigtool --hex-dump+ to convert any data into a hex-string:
    \begin{verbatim}
zolw@localhost:/tmp/test$ sigtool --hex-dump
How do I look in hex?
486f7720646f2049206c6f6f6b20696e206865783f0a
    \end{verbatim}

    \subsubsection{Wildcards}
    ClamAV supports the following extensions inside hex signatures:
    \begin{itemize}
	\item \verb+??+\\
	Match any byte.
	\item \verb+a?+\\
	Match a high nibble (the four high bits).\\ \textbf{IMPORTANT NOTE:}
	The nibble matching is only available in libclamav with the
	functionality level 17 and higher therefore please only use it with
	.ndb signatures followed by ":17" (MinEngineFunctionalityLevel,
	see \ref{ndb}).
	\item \verb+?a+\\
	Match a low nibble (the four low bits).
	\item \verb+*+\\
	Match any number of bytes.
	\item \verb+{n}+\\
	Match $n$ bytes.
	\item \verb+{-n}+\\
	Match $n$ or less bytes.
	\item \verb+{n-}+\\
	Match $n$ or more bytes.
	\item \verb+{n-m}+\\
	Match between $n$ and $m$ bytes ($m > n$).
	\item \verb+(aa|bb|cc|..)+\\
	Match aa or bb or cc..
	\item \verb+!(aa|bb|cc|..)+\\
	Match any byte except aa and bb and cc..
	\item \verb+HEXSIG[x-y]aa+ or \verb+aa[x-y]HEXSIG+\\
	Match aa anchored to a hex-signature, see
	\url{https://wwws.clamav.net/bugzilla/show_bug.cgi?id=776} for
	a discussion and examples.
    \end{itemize}
    The range signatures \verb+*+ and \verb+{}+ virtually separate
    a hex-signature into two parts, eg. \verb+aabbcc*bbaacc+ is treated
    as two sub-signatures \verb+aabbcc+ and \verb+bbaacc+ with any number
    of bytes between them. It's a requirement that each sub-signature
    includes a block of two static characters somewhere in its body.

    \subsubsection{Basic signature format}
    The simplest (and now deprecated) signature format is:
    \begin{verbatim}
MalwareName=HexSignature
    \end{verbatim}
    ClamAV will scan the entire file looking for HexSignature. All
    signatures of this type must be placed inside \verb+*.db+ files.

    \subsubsection{Extended signature format}\label{ndb}
    The extended signature format allows for specification of additional
    information such as a target file type, virus offset or engine version,
    making the detection more reliable. The format is:
    \begin{verbatim}
MalwareName:TargetType:Offset:HexSignature[:MinEngineFunctionalityLevel:[Max]]
    \end{verbatim}
    where \verb+TargetType+ is one of the following numbers specifying
    the type of the target file:
    \begin{itemize}
	\item 0 = any file
	\item 1 = Portable Executable, both 32- and 64-bit.
	\item 2 = file inside OLE2 container (e.g. image, embedded executable,
	VBA script). The OLE2 format is primarily used by MS Office and MSI
	installation files.
	\item 3 = HTML (normalized: whitespace transformed to spaces, tags/tag
	attributes normalized, all lowercase), Javascript is normalized too:
	all strings are normalized (hex encoding is decoded), numbers are
	parsed and normalized, local variables/function names are normalized
	to 'n001' format, argument to eval() is parsed as JS again,
	unescape() is handled, some simple JS packers are handled,
	output is whitespace normalized.
	\item 4 = Mail file
	\item 5 = Graphics
	\item 6 = ELF
	\item 7 = ASCII text file (normalized)
	\item 8 = Disassembler data
	\item 9 = Mach-O files
    \end{itemize}
    And	\verb+Offset+ is an asterisk or a decimal number \verb+n+ possibly
    combined with a special modifier:
    \begin{itemize}
	\item \verb+*+ = any
	\item \verb+n+ = absolute offset
	\item \verb+EOF-n+ = end of file minus \verb+n+ bytes
    \end{itemize}
    Signatures for PE, ELF and Mach-O files additionally support:
    \begin{itemize}
	\item \verb#EP+n# = entry point plus n bytes (\verb#EP+0# for \verb+EP+)
	\item \verb#EP-n# = entry point minus n bytes
	\item \verb#Sx+n# = start of section \verb+x+'s (counted from 0)
	data plus \verb+n+ bytes
	\item \verb#Sx-n# = start of section \verb+x+'s data minus \verb+n+ bytes
	\item \verb#SL+n# = start of last section plus \verb+n+ bytes
	\item \verb#SL-n# = start of last section minus \verb+n+ bytes
    \end{itemize}
    All the above offsets except \verb+*+ can be turned into
    \textbf{floating offsets} and represented as \verb+Offset,MaxShift+ where
    \verb+MaxShift+ is an unsigned integer. A floating offset will match every
    offset between \verb+Offset+ and \verb#Offset+MaxShift#, eg. \verb+10,5+
    will match all offsets from 10 to 15 and \verb#EP+n,y# will match all
    offsets from \verb#EP+n# to \verb#EP+n+y#. Versions of ClamAV older than
    0.91 will silently ignore the \verb+MaxShift+ extension and only use
    \verb+Offset+.\\

    \noindent
    All signatures in the extended format must be placed inside \verb+*.ndb+ files.

    \subsubsection{Logical signatures}\label{ndb}
    Logical signatures allow combining of multiple signatures in extended
    format using logical operators. They can provide both more detailed and
    flexible pattern matching. The logical sigs are stored inside \verb+*.ldb+
    files in the following format:
    \begin{verbatim}
SignatureName;TargetDescriptionBlock;LogicalExpression;Subsig0;
Subsig1;Subsig2;...
    \end{verbatim}
    where:
    \begin{itemize}
	\item \verb+TargetDescriptionBlock+ provides information about the
	engine and target file with comma separated \verb+Arg:Val+ pairs,
	currently (as of 0.95.1) only \verb+Target:X+ and \verb+Engine:X-Y+
	are supported.
	\item \verb+LogicalExpression+ specifies the logical expression
	describing the relationship between \verb+Subsig0...SubsigN+.\\
	\textbf{Basis clause:} 0,1,...,N decimal indexes are SUB-EXPRESSIONS
	representing \verb+Subsig0, Subsig1,...,SubsigN+ respectively.\\
	\textbf{Inductive clause:} if \verb+A+ and \verb+B+ are
	SUB-EXPRESSIONS and \verb+X, Y+ are decimal numbers then
	\verb+(A&B)+, \verb+(A|B)+, \verb+A=X+, \verb+A=X,Y+, \verb+A>X+,
	\verb+A>X,Y+, \verb+A<X+ and \verb+A<X,Y+ are SUB-EXPRESSIONS
	\item \verb+SubsigN+ is n-th subsignature in extended format possibly
	preceded with an offset. There can be specified up to 64 subsigs.
    \end{itemize}
    Modifiers for subexpressions:
    \begin{itemize}
	\item \verb+A=X+: If the SUB-EXPRESSION A refers to a single signature
	then this signature must get matched exactly X times; if it refers to
	a (logical) block of signatures then this block must generate exactly
	X matches (with any of its sigs).
	\item \verb+A=0+ specifies negation (signature or block of signatures
	cannot be matched)
	\item \verb+A=X,Y+: If the SUB-EXPRESSION A refers to a single signature
	then this signature must be matched exactly X times; if it refers to
	a (logical) block of signatures then this block must generate X matches
	and at least Y different signatures must get matched.
	\item \verb+A>X+: If the SUB-EXPRESSION A refers to a single signature
	then this signature must get matched more than X times; if it refers to
	a (logical) block of signatures then this block must generate more
	than X matches (with any of its sigs).
	\item \verb+A>X,Y+: If the SUB-EXPRESSION A refers to a single signature
	then this signature must get matched more than X times; if it refers to
	a (logical) block of signatures then this block must generate more than
	X matches and at least Y different signatures must be matched.
	\item \verb+A<X+ and \verb+A<X,Y+ as above with the change of "more"
	to "less".
    \end{itemize}
    Examples:
    \begin{verbatim}
Sig1;Target:0;(0&1&2&3)&(4|1);6b6f74656b;616c61;7a6f6c77;7374656
6616e;deadbeef

Sig2;Target:0;((0|1|2)>5,2)&(3|1);6b6f74656b;616c61;7a6f6c77;737
46566616e  

Sig3;Target:0;((0|1|2|3)=2)&(4|1);6b6f74656b;616c61;7a6f6c77;737
46566616e;deadbeef

Sig4;Target:1,Engine:18-20;((0|1)&(2|3))&4;EP+123:33c06834f04100
f2aef7d14951684cf04100e8110a00;S2+78:22??232c2d252229{-15}6e6573
(63|64)61706528;S+50:68efa311c3b9963cb1ee8e586d32aeb9043e;f9c58d
cf43987e4f519d629b103375;SL+550:6300680065005c0046006900
    \end{verbatim}

    \subsection{Signatures based on archive metadata}
    Signatures based on metadata inside archive files can provide an effective
    protection against malware that spreads via encrypted zip or rar
    archives. The format of a metadata signature is:
\begin{verbatim}
virname:encrypted:filename:normal size:csize:crc32:cmethod:fileno:max depth
\end{verbatim}
    where the corresponding fields are:
    \begin{itemize}
	\item Virus name
	\item Encryption flag (1 -- encrypted, 0 -- not encrypted)
	\item File name (this is a regular expression - * to ignore)
	\item Normal (uncompressed) size (* to ignore)
	\item Compressed size (* to ignore)
	\item CRC32 (* to ignore)
	\item Compression method (* to ignore)
	\item File position in archive (* to ignore)
	\item Maximum number of nested archives (* to ignore)
    \end{itemize}
    The database file should have the extension of \verb+.zmd+ or
    \verb+.rmd+ for zip or rar metadata respectively.

    \subsection{Whitelist databases}
    To whitelist a specific file use the MD5 signature format and place
    it inside a database file with the extension of \verb+.fp+.\\

    \noindent
    To whitelist a specific signature inside main.cvd add the following
    entry into daily.ign or a local file local.ign:
\begin{verbatim}
db_name:line_number:signature_name
\end{verbatim}

    \subsection{Signature names}
    ClamAV uses the following prefixes for signature names:
    \begin{itemize}
	\item \emph{Worm} for Internet worms
	\item \emph{Trojan} for backdoor programs
	\item \emph{Adware} for adware
	\item \emph{Flooder} for flooders
        \item \emph{HTML} for HTML files
        \item \emph{Email} for email messages
        \item \emph{IRC} for IRC trojans
	\item \emph{JS} for Java Script malware
	\item \emph{PHP} for PHP malware
	\item \emph{ASP} for ASP malware
	\item \emph{VBS} for VBS malware
	\item \emph{BAT} for BAT malware
	\item \emph{W97M}, \emph{W2000M} for Word macro viruses
	\item \emph{X97M}, \emph{X2000M} for Excel macro viruses
	\item \emph{O97M}, \emph{O2000M} for generic Office macro viruses
	\item \emph{DoS} for Denial of Service attack software
	\item \emph{DOS} for old DOS malware
	\item \emph{Exploit} for popular exploits
	\item \emph{VirTool} for virus construction kits
	\item \emph{Dialer} for dialers
	\item \emph{Joke} for hoaxes
    \end{itemize}
    Important rules of the naming convention:
    \begin{itemize}
	\item always use a -zippwd suffix in the malware name for signatures of	      type zmd,
	\item always use a -rarpwd suffix in the malware name for signatures
	      of type rmd,
	\item only use alphanumeric characters, dash (-), dot (.), underscores
	      (\_) in malware names, never use space, apostrophe or quote mark.
    \end{itemize}

    \section{Special files}

    \subsection{HTML}
    ClamAV contains a special HTML normalisation code which helps to detect
    HTML exploits. Running \verb+sigtool --html-normalise+ on a HTML file
    should generate the following files:
    \begin{itemize}
	\item nocomment.html - the file is normalized, lower-case, with all
	comments and superflous white space removed
	\item notags.html - as above but with all HTML tags removed
    \end{itemize}
    The code automatically decodes JScript.encode parts and char ref's (e.g.
    \verb+&#102;+). You need to create a signature against one of the created
    files. To eliminate potential false positive alerts the target type should
    be set to 3.

    \subsection{Text files}
    Similarly to HTML all ASCII text files get normalized (converted
    to lower-case, all superflous white space and control characters removed,
    etc.) before scanning. Use \verb+clamscan --leave-temps+ to obtain
    a normalized file then create a signature with the target type 7.

    \subsection{Compressed Portable Executable files}
    If the file is compressed with UPX, FSG, Petite or other PE packer
    supported by libclamav, run \verb+clamscan+ with
    \verb+--debug --leave-temps+. Example output for a FSG compressed file:
    \begin{verbatim}
LibClamAV debug: UPX/FSG/MEW: empty section found - assuming compression
LibClamAV debug: FSG: found old EP @119e0
LibClamAV debug: FSG: Unpacked and rebuilt executable saved in
/tmp/clamav-f592b20f9329ac1c91f0e12137bcce6c
    \end{verbatim}
    Next create a type 1 signature for \verb+/tmp/clamav-f592b20f9329ac1c91f0e12137bcce6c+

\end{document}
